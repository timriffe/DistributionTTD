\documentclass{article}
\usepackage{amsmath}
\usepackage{caption}
\usepackage{placeins}
\usepackage{graphicx}
\usepackage{subcaption}
\usepackage{tikz}
%\usepackage[active,tightpage]{preview}
\usepackage{natbib}
\bibpunct{(}{)}{,}{a}{}{;} 
\usepackage{url}
\usepackage{nth}
% for the d in integrals
\newcommand{\dd}{\; \mathrm{d}}
\newcommand{\tc}{\quad\quad\text{,}}
\newcommand{\tp}{\quad\quad\text{.}}

\begin{document}

\title{Distributional aspects of time to death in human populations}
\author{Tim Riffe}
\maketitle

\section{Background}

Typically demographers are satisfied to summarize the remaining lifetime for age
groups using a mean, $e(a)$, which is however not an omnibus descriptor of time
to death. Other measures are known to be useful measures of longevity, such as
the modal age at death or the median, and demographers also have a battery of
indicators for lifespan variability or entropy. One aspect in common of all
these indicators except for remaining life expectancy is that they refer to the
age distribution of mortality in a snapshot of a stationary population or else
the age at death distribution of a newborn cohort under constant mortality
conditions. These measures are not typically made conditional on survival to
later ages, i.e., demographers too seldom consider the properties of the
distribution of remaining lifetime given survival to a particular age. All of
the above measures can be reworked as functions of remaining lifetime (i.e. the
modal remaining lifespan), and indeed we may fruitfully define some standard
statistical descriptors of shape to the distribution of remaining lifespans. In
this paper, we aim to provide some elementary definitions and apply these to
human populations. As an example of utility of this perspective we offer a
selection of simple heuristics for late life decisions based on these
distributional measures.

\section{Definitions}

Remaining life expectancy conditional on survial to age $a$ is defined as
\begin{equation}
e(a) = \frac{\int_{y=0}^\infty l(a+y) \dd y}{l(a)} \tp
\end{equation}

Say lifespans for a given birth cohort are measured with the random variable,
$X$, with distribution $d(X)$, which is identical to the $d_x$
column of the lifetable if a radix of 1 is used. In othe words, the
distribution of lifespans for newborns is identical to the distribution of
age at death in the stationary population. We are interested in $f(X-a ~|~ X \ge
a)$, which we denote using $f(y|a)$, and which is defined as:
\begin{equation}
\label{eq:fya}
f(y|a) = \mu(a+y) \frac{l(a+y)}{l(a)} \tc
\end{equation}
i.e., the probability of surviving to and dying at age $a+y$ given survival to
age $a$. \eqref{eq:fya} can also be used to calculate remaining life expectancy:
\begin{equation}
e(a) = \int _{y=0}^\infty y f(y|a) \dd y \tp
\end{equation}

Demographers make less frequent
reference to $f(y|a)$, which is however useful for decomposing
demographic counts into remaining lifetime classes. The empirical distribution
of $f(y|a)$ is a worthy topic of study, as it bears heavily on many late-life
decisions, such as bequesting or moving into a care residence, and possibly also
on the ethical pondering of public policies, such as mandatory retirement ages
or pensions. The conditional distribution of remaining lifetimes can be
described empirically, or more parsimoniously using the standard tools of 
statistics (central moments) used to describe distributions: the variance,
skewness and kurtosis.

The $n^{th}$ momet of $f(y|a)$, $\eta_n(y|a)$ is defined as: 
\begin{align}
\eta _n(y|a) =& \frac{1}{l(a)} \int_{y=0}^\infty (y-e(a))^n \mu (a+y) l(a+y) \dd
y 
\intertext{or just}
\eta _n(y|a) =&  \int_{y=0}^\infty (y-e(a))^n f(y|a) \dd y \tc
\end{align}
where $n=2$ gives the variance, $\sigma^2(y|a)$, $n=3$ gives the skewness and
$n=4$ gives the kurtosis. 

The coefficient of variation of remaining lifespan, $CV(y|a)$ is then simply
\begin{equation}
CV(y|a) = \frac{\sqrt{\sigma^2(y|a)}}{e(a)}
\end{equation}

\noindent $CV(y|a)$ is dimensionless and comparable over age, and its reciprocal
can be thought of as a signal to noise ratio of one's likely remaining lifespan,
assuming a constant mortality pattern in ages higher than $a$.

\section{Observed patterns}
The above definitions are perhaps too clean to be analytically useful outside
of idealized parametric mortality settings. We wish to examine some patterns
of these distributional properties. For instance, is there a crossover age in
skewness, how does this vary over time and between populations?

Likewise, what is the age where the variance in remaining lifespan hits it's
minimum, and how stable is this age over time and between populations?

\section{Heuristics}
Ceteris paribus (i.e., in a homogenous population and where investment
specifics do not play in), negative skewness suggests that we buy an (actuarily
fair) annuity and positive skewness implies that the odds of us surviving to get 
the money back are less than unity?

What would the ethicist say about the just span of retirement? The variance of
time to death is not monotonic, but instead decreases until some late age and
then increases again. Could a just age of mandatory retirement be the age where
the variance in remaining lifespan is at a minimum? To equalize the duration of
retirement between individuals is at first glance a step toward equality, but
there is no anecessary or best way to determine this duration, and there is also
no necessary way to adjust this duration as a function of lifespan itself. In
other words, if the desired duration of retirement is 20 years, perhaps the
meaning of 20 years is greater for an individual with a shorter lifespan and
lesser for a longer total lifespan.

Can the age of minimum variance be used as another sort of demographic threshold
for ageing or dependency?


\end{document}
